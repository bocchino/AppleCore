\documentclass[10pt]{article}

\usepackage{comment}
\usepackage{fullpage}
\usepackage{multicol}
\usepackage{amsmath}

\usepackage{fouriernc}

\setlength{\parskip}{1.5ex plus 0.5ex minus 0.5ex}

\newcommand{\ithead}[1]{\noindent\textit{#1.}}
\newcommand{\bfhead}[1]{\noindent\textbf{#1.}}
\newcommand{\kwd}[1]{\texttt{#1}}
\newcommand{\bs}{\textbackslash}
\newcommand{\nonterm}[1]{\ensuremath{\mbox{\textit{#1}}}}
\newcommand{\opt}[1]{\textit{[} #1 \textit{]}}
\newcommand{\group}[1]{\textit{(} #1 \textit{)}}

\newcommand{\shl}[0]{\texttt{<}\texttt{<}}
\newcommand{\shr}[0]{\texttt{>}\texttt{>}}

\newcommand{\vartype}[0]{\kwd{:} \group{\nonterm{size} \opt{\kwd{S}} | \kwd{@}}}


\begin{document}

\title{\bfseries{The AppleCore Virtual Machine Specification, v1.0}}
%
\author{Robert L. Bocchino Jr.\\
Pittsburgh, PA}

\maketitle

\section{Introduction and Rationale}

The AppleCore Virtual Machine (AVM) is a virtual machine architecture
for the AppleCore programming language.  Its purposes are:
%
\begin{enumerate}
%
\item To provide a virtual instruction set that closely models the
  stack semantics of the AppleCore language.  The virtual instruction
  set can be used as a single intermediate representation for
  AppleCore source programs that is translated to either 6502 code or
  AVM bytecode.
%
\item To provide a bytecode representation that stores AppleCore
  programs in a very compact way (using about 2.5x fewer bytes than
  native code).  This allows larger programs to be stored in memory.
  However, runtime interpretation of AppleCore bytecode is slower
  than running native code; exactly how much slower remains to be
  seen.
%
\end{enumerate}

\section{Integration with 6502 Code}

The AVM instruction set is designed to represent AppleCore functions
in a way that interoperates with 6502 code.  In particular, caller
code should have no way of knowing whether a called function is
implemented in 6502 or AVM instructions.  Also, interpreted AVM
instructions must be able to call native code, because the AppleCore
specification says that calls to AppleCore functions and regular
assembly languages functions should be interchangeable.

These requirements met as follows.  As is typical for virtual machine
architectures on the Apple (for example, see Steve Wozniak's Sweet-16
interpreter), interpretation starts when the 6502 code does a
\kwd{JSR} to the interpreter.  The interpreter pulls the ``return
address'' off the stack and uses it to determine where to start
interpreting bytecode: all the bytes following the \kwd{JSR}
instruction are interpreted as AVM code, up to and including the first
\kwd{RAF} (Return from AppleCore Function) encountered.

In code translated from AppleCore source, each function starts with a
\kwd{JSR} to the interpreter, and all statements in the function are
translated to AVM code.  That way, the function works normally when
called from normal 6502 code.  Also, subroutine calls within AVM
code work the same way regardless of whether the callee function
contains AVM or native code.

\section{Instruction Set}

The AVM instruction set has three kinds of instructions: unsized instructions, sized
instructions, and signed instructions.

\subsection{Unsized Instructions}

\bfhead{BRK (Break, Opcode \$00):}
%
Causes the AVM interpreter to execute a 6502 \kwd{BRK} instruction.

\bfhead{BRF (Branch on Result False, Opcode \$01):}
%
Causes the AVM interpreter to pull a single byte off the AppleCore
program stack.  If the byte evaluates to true (i.e., has its low
bit set, see Section 4.1 of the AppleCore Language Specification),
then interpretation continues with the instruction located three
bytes after this one.  Otherwise, control branches to the address
given by the next two bytes.

\bfhead{BRU (Branch Unconditionally, Opcode \$02):}
%
Causes control to branch to the address given by the two bytes
following the instruction opcode.

\bfhead{CFD (Call Function Direct, Opcode \$03):}
%
Causes the AVM interpreter to execute a \kwd{JSR} to the address
given by the two bytes following the instruction opcode.  On
return from the \kwd{JSR}, execution resumes with the third byte
following the instruction opcode.

\bfhead{CFI (Call Function Indirect, Opcode \$04):}
%
Causes the AVM interpreter to pull two bytes off the stack
and do a \kwd{JSR} to code that branches to the address given
by those two bytes.  On return from the \kwd{JSR}, execution
resumes with the byte following the instruction opcode.

\bfhead{NOP (No Operation, Opcode \$EA):}
%
Causes the AVM interpreter to skip the instruction and
continue execution with the byte following the instruction
opcode.

\subsection{Sized Instructions}

TODO

\subsection{Signed Instructions}

TODO

\subsection{Unused Opcodes}

TODO


\end{document}
