
\documentclass[10pt]{article}

\usepackage{comment}
\usepackage{fullpage}
\usepackage{multicol}
\usepackage{amsmath}

\usepackage{fouriernc}

\setlength{\parskip}{1.5ex plus 0.5ex minus 0.5ex}

\newcommand{\ithead}[1]{\noindent\textit{#1.}}
\newcommand{\bfhead}[1]{\noindent\textbf{#1.}}
\newcommand{\kwd}[1]{\texttt{#1}}
\newcommand{\bs}{\textbackslash}
\newcommand{\nonterm}[1]{\ensuremath{\mbox{\textit{#1}}}}
\newcommand{\opt}[1]{\textit{[} #1 \textit{]}}
\newcommand{\group}[1]{\textit{(} #1 \textit{)}}

\newcommand{\shl}[0]{\texttt{<}\texttt{<}}
\newcommand{\shr}[0]{\texttt{>}\texttt{>}}

\newcommand{\vartype}[0]{\kwd{:} \group{\nonterm{size} \opt{\kwd{S}} | \kwd{@}}}


\begin{document}

\title{\bfseries{The AppleCore Virtual Machine Specification, v1.0}}
%
\author{Robert L. Bocchino Jr.\\
Pittsburgh, PA}

\maketitle

\section{Introduction and Rationale}

The AppleCore Virtual Machine (AVM) is a virtual machine architecture
for the AppleCore programming language.  Its purposes are:
%
\begin{enumerate}
%
\item To provide a virtual instruction set that closely models the
  stack semantics of the AppleCore language.  The virtual instruction
  set can be used as a single intermediate representation for
  AppleCore source programs that is translated to either 6502 code or
  AVM bytecode.
%
\item To provide a bytecode representation that stores AppleCore
  programs in a very compact way (using about 2.5x fewer bytes than
  native code).  This allows larger programs to be stored in memory.
  However, runtime interpretation of AppleCore bytecode is slower
  than running native code; exactly how much slower remains to be
  seen.
%
\end{enumerate}

\section{Integration with 6502 Code}

The AVM instruction set is designed to represent AppleCore functions
in a way that interoperates with 6502 code.  In particular, caller
code should have no way of knowing whether a called function is
implemented in 6502 or AVM instructions.  Also, interpreted AVM
instructions must be able to call native code, because the AppleCore
specification says that calls to AppleCore functions and regular
assembly language functions are interchangeable.

These requirements are met as follows.  As is typical for virtual machine
architectures on the Apple (for example, see Steve Wozniak's Sweet-16
interpreter), interpretation starts when the 6502 code does a
\kwd{JSR} to the interpreter.  The interpreter pulls the ``return
address'' off the Apple II stack and uses it to determine where to start
interpreting bytecode: all the bytes following the \kwd{JSR}
instruction are interpreted as AVM code, up to and including the first
\kwd{RAF} (Return from AppleCore Function) encountered.
Executing \kwd{RAF} returns 6502 control to the return
address that was on the stack when the interpreter was
invoked.

In code translated from AppleCore source, each function starts with a
\kwd{JSR} to the interpreter, and all statements in the function are
translated to AVM code.  That way, the function works normally when
called from normal 6502 code.  Also, subroutine calls within AVM
code work the same way regardless of whether the callee function
contains AVM or native code.
The call to the interpreter also stores the callee's frame pointer
on the stack, as specified in Section 5.4 of the AppleCore
language specification.

\section{Instruction Set}

The AVM instruction set has three kinds of instructions: unsized instructions, sized
instructions, and signed instructions.

\subsection{Unsized Instructions}

\bfhead{BRK (Break, Opcode \$00):}
%
Causes the AVM interpreter to execute a 6502 \kwd{BRK} instruction.

\bfhead{BRF (Branch on Result False, Opcode \$01):}
%
Causes the AVM interpreter to pull a single byte off the AppleCore
program stack.  If the byte evaluates to true (i.e., has its low
bit set, see Section 4.1 of the AppleCore Language Specification),
then interpretation continues with the instruction located three
bytes after this one.  Otherwise, control branches to the address
given by the next two bytes.

\bfhead{BRU (Branch Unconditionally, Opcode \$02):}
%
Causes control to branch to the address given by the two bytes
following the instruction opcode.

\bfhead{CFD (Call Function Direct, Opcode \$03):}
%
Causes the AVM interpreter to execute a \kwd{JSR} to the address
given by the two bytes following the instruction opcode.  On
return from the \kwd{JSR}, execution resumes with the third byte
following the instruction opcode.

\bfhead{CFI (Call Function Indirect, Opcode \$04):}
%
Causes the AVM interpreter to pull two bytes off the stack
and do a \kwd{JSR} to code that branches to the address given
by those two bytes.  On return from the \kwd{JSR}, execution
resumes with the byte following the instruction opcode.

\bfhead{NOP (No Operation, Opcode \$EA):}
%
Causes the AVM interpreter to skip the instruction and
continue execution with the byte following the instruction
opcode.

\subsection{Sized Instructions}

Every sized instruction has a one-byte unsigned size argument.
If the value of the size argument is between 0 and 6 inclusive,
then the size argument is stored in the low-order 3 bits of the
opcode; otherwise the low-order 3 bits are all
ones and the size is stored in the byte immediately following
the opcode.  Instructions \kwd{MTV}, \kwd{PHC}, and \kwd{VTM}
each have an additional
argument which is stored in the byte or bytes immediately
following the one or two bytes representing the opcode and
size.

\bfhead{ADD (Add, Opcodes \$08--\$0F):} Pull two \emph{size}-byte
values off the stack, add the values ignoring overflow, and push the
result on the stack.

\bfhead{AND (Bitwise And, Opcodes \$10--\$17):} Pull two
\emph{size}-byte values off the stack, compute the bitwise and of the
values, and push the result on the stack.

\bfhead{DEC (Decrement Memory, Opcodes \$18--\$1F):} Pull a two-byte
address off the stack and decrement the \emph{size}- byte value stored
at that address in memory.  Push the result on the stack.

\bfhead{DSP (Decrease Stack Pointer, Opcodes \$20--\$27):}
Decrease the AppleCore program stack pointer by \emph{size}.

\bfhead{INC (Increment Memory, Opcodes \$28--2F):} Pull a two-byte
address off the stack and increment the \emph{size}-byte value stored
at that address in memory.  Push the result on the stack.

\bfhead{ISP (Increase Stack Pointer, Opcodes \$30--\$37):}
Decrease the AppleCore program stack pointer by \emph{size}.

\bfhead{MTS (Memory To Stack, Opcodes \$38--\$3F):}
Pull a two-byte address off the stack. Push the \emph{size}
bytes in memory starting at that address on the stack.

\bfhead{MTV (Memory To Variable, Opcodes \$40--\$47):}
Treat the byte following the opcode and size as a zero-page
address.  Read the value from that address and store it
at address $\kwd{FP}+\nonterm{size}$, where \kwd{FP} is the
AppleCore frame pointer.

\bfhead{NEG (Arithmetic Negation, Opcodes \$48--\$4F):} Pull a
\emph{size}-byte value off the stack, negate the value (i.e., form the
two's complement), and push the result on the stack.

\bfhead{NOT (Logical Negation, Opcodes \$50--\$57):} Pull a
\emph{size}-byte value off the stack, logically negate the bits of the
value, and push the result on the stack.

\bfhead{ORL (Or Logical, Opcodes \$58--\$5F):} Pull two
\emph{size}-byte values off the stack, form the logical or of the two
values, and push the result on the stack.

\bfhead{ORX (Or Exclusive, Opcodes \$60--\$67):} Pull two
\emph{size}-byte values off the stack, form the logical exclusive or
of the two values, and push the result on the stack.

\bfhead{PHC (Push Constant, Opcodes \$68--\$6F):}
Interpret the \emph{size} bytes following the opcode and size
as a \emph{size}-byte constant.  Push that constant on the
stack.

\bfhead{PVA (Push Variable Address, Opcodes \$70--\$77):}
Push the two-byte address given by $\kwd{FP}+\nonterm{size}$ on
the stack.

\bfhead{RAF (Return from AppleCore Function, Opcodes \$78--\$7F):}
Pull \emph{size} bytes off the stack.  Set the stack pointer to the
frame pointer.  Restore the caller's frame pointer.  Push the
\emph{size} bytes on the stack.  Return control to the address on the
6502 stack.

\bfhead{SHL (Shift Left, Opcodes \$80--\$87):} Pull a single-byte
unsigned
shift amount off the stack.  Pull a \emph{size}-byte value off the
stack, shift it left by the shift amount, and push the result on the
stack.

\bfhead{STM (Stack To Memory, Opcodes \$87--\$8F):}
Pull a two-byte address off the stack.  Pull a \emph{size}-byte
value off the stack and store it in memory starting at the
address.

\bfhead{SUB (Subtract, Opcodes \$90--\$97):}
Pull two \emph{size}-byte values off the stack, subtract the
second one from the first one, and push the result on the stack.

\bfhead{TEQ (Test Equal, Opcodes \$98--\$9F):}
Pull two \emph{size}-byte values off the stack. Push 1 on the stack
if they are equal, 0 if they are not equal.

\bfhead{VTM (Variable To Memory, Opcodes \$A0--\$A7):}
Treat the byte following the opcode and size as a zero-page
address.  Read the value at address $\kwd{FP}+\nonterm{size}$, 
where \kwd{FP} is the AppleCore frame pointer, and store
it at the zero-page address.

\subsection{Signed Instructions}

TODO

\subsection{Unused Opcodes}

TODO

\section{Table of Opcodes}

TODO


\end{document}
