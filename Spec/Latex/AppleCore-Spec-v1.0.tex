\documentclass[10pt]{article}

\usepackage{comment}
\usepackage{fullpage}
\usepackage{multicol}
\usepackage{amsmath}

\usepackage{fouriernc}

%\setlength{\oddsidemargin}{0in}
%\setlength{\evensidemargin}{0in}
%\setlength{\topmargin}{0in}
%\setlength{\botmargin}{0in}
%\setlength{\textwidth}{6.5in}
\setlength{\parskip}{1.5ex plus 0.5ex minus 0.5ex}

%\pagestyle{myheadings}

\newcommand{\ithead}[1]{\noindent\textit{#1}.}
\newcommand{\bfhead}[1]{\noindent\textbf{#1}.}
\newcommand{\kwd}[1]{\texttt{#1}}
\newcommand{\bs}{\textbackslash}
\newcommand{\nonterm}[1]{\ensuremath{\mbox{\textit{#1}}}}
\newcommand{\opt}[1]{\textit{[} #1 \textit{]}}
%\newcommand{\opt}[1]{#1$?$}
\newcommand{\group}[1]{\textit{(} #1 \textit{)}}

\newcommand{\shl}[0]{\texttt{<}\texttt{<}}
\newcommand{\shr}[0]{\texttt{>}\texttt{>}}

\newcommand{\vartype}[0]{\kwd{:} \group{\nonterm{size} \opt{\kwd{S}} | \kwd{@}}}


\begin{document}

\title{\bfseries{The AppleCore Language Specification, v1.0}}
%
\author{Robert L. Bocchino Jr.\\
Pittsburgh, PA}
%
\date{Revised December 4, 2011}

\maketitle

TODO

\section{Introduction and Rationale}

TODO

\section{Lexical Structure}

Before parsing a source file, the source text is separated into
\emph{tokens}.  There are four classes of tokens: identifiers,
keywords, constants, and symbols.

\subsection{White Space}
\label{sec:lexical:white-space}

White space consists of any sequence of the following characters:
space (ASCII SP, value \$20), newline (ASCII NL, value \$0A), carriage
return (ASCII CR, value \$0D), and horizontal tab (ASCII HT, value
\$09).  Whitespace is always useful in formatting readable programs,
but it is syntactically irrelevant except in the following two cases:
%
\begin{enumerate}
%
\item \ithead{Separation of tokens.} When a whitespace character or end of
file appears immediately after a non-whitespace character, that
signifies the end of a token.
%
\item \ithead{End of line.} An end-of-line sequence (\kwd{EOL}) is a
  comment terminator (see Section~\ref{sec:lexical:comments}).  As
  usual, the definition of \kwd{EOL} is platform-dependent: on the
  Apple II it is CR, on UNIX it is NL, and on Windows it is CR
  followed by NL.
%
\end{enumerate}
%
The lexer may also count source lines (using \kwd{EOL}) to provide
line numbers for error messages.

\subsection{Identifiers}\label{Identifiers}

An identifier is a sequence of letters (including the underscore
character) and digits that is not a keyword.  The first character must
be a letter.  Upper and lower case letters are distinct.

\subsection{Keywords}

The following sequences of characters are reserved for use as keywords
and may not be used as identifiers:
\begin{ttfamily}
\begin{center}
\begin{tabular}{l l l l}
AND & CONST & DATA & DECR \\
%
ELSE & FN & IF & INCLUDE \\
%
INCR & NOT & OR & RETURN \\
%
SET & VAR & XOR & WHILE \\
\end{tabular}
\end{center}
\end{ttfamily}

Following Apple II tradition (mostly because the original Apple II had
no support for lower-case letters), AppleCore keywords are uppercase.
That means that the same words in lowercase (or a combination of
upper- and lowercase) are not recognized as keywords, so those
sequences of characters are availble for use as identifiers.

\subsection{Constants}
\label{sec:lexical:constants}

There are three types of constants: integer constants, string
constants, and character constants.  

\bfhead{Integer constants.}  Integer constants may be written in
decimal or hexadecimal form.  A decimal integer constant consists of
one or more decimal digits \kwd{0} through \kwd{9}.  For example,
\kwd{1} and \kwd{123} are valid integer constants in decimal form.  A
hexadecimal integer constant consists of a dollar sign \kwd{\$}
followed by by one or more hexadecimal digits \kwd{0} through \kwd{9}
or \kwd{A} through \kwd{F}.  For example, \kwd{\$1}, \kwd{\$A}, and
\kwd{\$00FF} are valid integer constants in hexadecimal form.

\bfhead{String constants.} A string constant is a sequence of
characters enclosed in double quotes (\kwd{"..."}).  A string constant
represents a sequence of ASCII characters, one for each character
appearing in the double-quotes, except that the character sequence
\kwd{\bs\$} has the special meaning described below.

Since not all ASCII characters have printable representations, an
\emph{escape sequence} may be used to represent an arbitrary ASCII
value (printable or non-printable).  An escape sequence consists of a
backslash \kwd{\bs} followed by a dollar sign \kwd{\$} and two
hexadecimal digits.  The whole sequence represents the single
character with the ASCII value given by the digits.  For example, the
string constant
%
$$\kwd{"Hello, world!\bs\$0D"}$$
%
represents a null-terminated string consisting of the characters
\kwd{Hello, world!} followed by a CR character.  Similarly, the quote
character \kwd{"} can be embedded in a string constant with the
sequence \kwd{\bs\$22}.

\bfhead{Character constants.} A character constant consists of a
single-quote character \kwd{'}, followed by a printable ASCII
character, followed by another single-quote character.  It represents
the ASCII value associated with the character.  For example, the
constant \kwd{'A'} represents the value $\$41$.

\subsection{Symbols}

AppleCore uses the symbols shown in Figure~\ref{fig:symbols}, each of
which is a separate token.

\begin{figure}[th]
\begin{center}
\begin{tabular}{c l}
\textbf{Symbol} & \textbf{Meaning} \\
%
\kwd{@} & Denotes the address of a variable \\
%
\kwd{\^} & Denotes a 6502 register expression \\
\kwd{*}  & Multiplication \\
\kwd{/}  & Division \\
\kwd{+}  & Addition \\
\kwd{-}  & Negation (as unary operator); subtraction (as binary
operator) \\
\kwd{<<} & Left shift \\
\kwd{>>} & Right shift \\
\kwd{>=} & Greater than or equal to \\
\kwd{<=} & Less than or equal to \\
\kwd{>}  & Greater than \\
\kwd{<}  & Less than \\
\kwd{=}  & Equal to; assignment \\
\kwd{(} and \kwd{)} & Encloses function parameters and arguments, parenthesized expressions
\\
\kwd{\{} and \kwd{\}} & Encloses statement blocks \\
\kwd{[} and \kwd{]} & Pointer dereference \\
\kwd{;} & Terminates declarations and statements \\
\kwd{:} & Separates variable declaration from size \\
\kwd{,} & Separates function parameters and arguments; separates index from size in dereference \\
\end{tabular}
\end{center}
\caption{Symbols used in AppleCore syntax.}
\label{fig:symbols}
\end{figure}

\subsection{Comments}
\label{sec:lexical:comments}

The character \kwd{\#} in the source file indicates a comment; all
text to the next end of line (or end of file, if there is no end of
line) are ignored by the lexer.  Multiline comments are indicated by
preceding each line with \kwd{\#}.

\section{Syntax}
\label{sec:syntax}

The syntax description below uses the following conventions:
%
\begin{itemize}
%
\item The symbol $^*$ denotes zero or more instances of the entity
  preceding it.
%
\item Italicized parentheses \group{} group the enclosed symbols and
  do not denote program text.
%
\item Italicized brackets \opt{} signify that the enclosed symbols are
  optional (i.e., they may occur zero or one time).  They do not
  denote program text.
%
\item The nonterminal \nonterm{identifier} stands for any identifier
  as defined in Section \ref{Identifiers}.
%
\item The nonterminals \nonterm{integer-const},
  \nonterm{string-const}, and \nonterm{char-const} stand for integer,
  string, and character constants as defined in
  Section~\ref{sec:lexical:constants}.
%
\item The nonterminal \nonterm{size} refers to an integer constant
  whose value is between 0 and 255 (inclusive).
%
\item Text and symbols in \kwd{typewriter} font (including
  non-italicized parentheses and brackets) denote literal program
  text.
%
\end{itemize}

\subsection{Source Files}
\label{sec:syntax:source-files}

The basic syntactic unit of an AppleCore program is a \emph{source
  file}, i.e., an input file provided to the AppleCore compiler for
compilation.  The compiler translates the source file into an assembly
file which is then linked with other assembly files as discussed in
Section~\ref{sec:semantics:compilation} to form a complete executable
program.

An AppleCore source file is given by zero or more
declarations:
%
$$\nonterm{source-file} ::= \nonterm{decl}^*$$
%
A declaration is a constant declaration, a data declaration, a
variable declaration, a function declaration, or an include
declaration:
%
$$\nonterm{decl} ::= \mbox{\nonterm{const-decl} $|$
  \nonterm{data-decl} $|$ \nonterm{var-decl} $|$ \nonterm{fn-decl}
    $|$ \nonterm{include-decl}}$$

\subsection{Declarations}
\label{sec:syntax:declarations}

\bfhead{Constant declarations.} A constant declaration consists of the
keyword \kwd{CONST}, an identifier, an optional expression, and a
terminating semicolon:
%
$$\nonterm{const-decl} ::= \mbox{\kwd{CONST} \nonterm{identifier}
  \opt{\nonterm{expr}} \kwd{;}}$$

\bfhead{Data declarations.} A data declaration consists of the keyword
\kwd{DATA}, an optional identifier representing a label for the data,
an expression or a string constant, and a terminating semicolon.  A
backslash may optionally follow the string constant, indicating that
the string is unterminated (see Section~\ref{}).
%
$$\nonterm{const-decl} ::= \mbox{\kwd{DATA} \opt{\nonterm{identifier}}
  \group{\nonterm{expr} $|$ \group{\nonterm{string-const} \opt{\bs}}}
  \kwd{;}}$$

\bfhead{Variable declarations.} A variable declaration consists of the
keyword \kwd{VAR}, an identifier representing the variable name, a
signed or unsigned size, an optional initializer expression, and a
terminating semicolon:
%
$$\nonterm{var-decl} ::= \mbox{\kwd{VAR} \nonterm{identifier} \kwd{:}
  \nonterm{size} \opt{\kwd{S}} \opt{\kwd{=} \nonterm{expr}} \kwd{;}}$$

\bfhead{Function declarations.} A function declaration consists of the
keyword \kwd{FN}, an optional signed or unsigned size, an identifier
representing the function name, the function parameters enclosed in
parentheses, and the function body:
%
$$\nonterm{fn-decl} ::= \mbox{\kwd{FN} \opt{\kwd{:} \nonterm{size}
    \opt{\kwd{S}}} \nonterm{identifier} \kwd{(} \nonterm{fn-params}
  \kwd{)} \nonterm{fn-body}}$$
%
The function parameters are a comma-separated list of zero or more
parameters:
%
$$\nonterm{fn-params} ::= \mbox{\opt{\nonterm{fn-param} \group{\kwd{,}
    \nonterm{fn-param}}$^*$}}$$
%
A parameter consists of an identifier representing the parameter name
and a size:
%
$$\nonterm{fn-param} ::= \mbox{\nonterm{identifier} \kwd{:}
  \nonterm{size} \opt{\kwd{S}}}$$
%
A function body is either zero or more variable declarations and
statements enclosed in braces, or a semicolon indicating an externally
defined function:
%
$$\nonterm{fn-body} ::= \mbox{\kwd{\{} \nonterm{var-decl}$^*$
  \nonterm{stmt}$^*$ \kwd{\}} $|$ \kwd{;}}$$

\bfhead{Include declarations.} An include declaration consists of the
keyword \kwd{INCLUDE}, a string constant, and a terminating semicolon:
%
$$\nonterm{include-decl} ::= \mbox{\kwd{INCLUDE}
  \nonterm{string-const} \kwd{;}}$$
%

\subsection{Statements}
\label{sec:syntax:statements}

A statement is an if statement, a while statement, a return statement,
a block statement, or an expression statement:
%
$$\nonterm{stmt} ::= \mbox{\nonterm{if-stmt} $|$ \nonterm{while-stmt}
  $|$ \nonterm{return-stmt} $|$ \nonterm{block-stmt} $|$
  \nonterm{expr-stmt}}$$
%
\bfhead{If statements.} An if statement consists of the keyword
\kwd{IF}, a conditional expression enclosed in parentheses, a
statement to execute if the condition is true, and optionally the
keyword \kwd{ELSE} followed by a statement to execute if the condition
is false:
%
$$\nonterm{if-stmt} ::= \mbox{\kwd{IF} \kwd{(} \nonterm{expr} \kwd{)}
  \nonterm{stmt} \opt{\kwd{ELSE} \nonterm{stmt}}}$$
%

\bfhead{While statements.} A while statement consists of the keyword
\kwd{WHILE}, a test expression enclosed in parentheses, and a
statement to execute as long as the condition is true:
%
$$\nonterm{while-stmt} ::= \mbox{\kwd{WHILE} \kwd{(} \nonterm{expr}
  \kwd{)} \nonterm{stmt}}$$
%

\bfhead{Return statements.} A return statement consists of the keyword
\kwd{RETURN}, an optional expression, and a terminating semicolon:
%
$$\nonterm{return-stmt} ::= \mbox{\kwd{RETURN} \opt{\nonterm{expr}}
    \kwd{;}}$$
%

\bfhead{Block statements.} A block statement consists of zero or more
statements enclosed in braces:
%
$$\nonterm{block-stmt} ::= \mbox{\kwd{\{} \nonterm{stmt}$^*$
  \kwd{\}}}$$
%
\bfhead{Expresson statements.} An expression statement consists of an
expression followed by a terminating semicolon:
%
$$\nonterm{expr-stmt} ::= \mbox{\nonterm{expr} \kwd{;}}$$
%

\subsection{Expressions}
\label{sec:syntax:expressions}

An expression is an lvalue expression, a call expression, a set
expression, a binary operation expression, a unary operation
expression, a numeric constant, or a parentheses expression:
%
$$\nonterm{expr} ::= \mbox{\nonterm{lvalue-expr} $|$
  \nonterm{call-expr} $|$ \nonterm{set-expr} $|$ \nonterm{binop-expr}
  $|$ \nonterm{unop-expr} $|$ \nonterm{numeric-const} $|$
  \nonterm{parens-expr}}$$
%
An lvalue expression is an expression that may appear on the
right-hand side of a set expression.  It is an identifier, an indexed
expression, or a register expression:
%
$$\nonterm{lvalue-expr} ::= \mbox{\nonterm{identifier} $|$
  \nonterm{indexed-expr} $|$ \nonterm{register-expr}}$$
%
A numeric constant is an integer or character constant:
%
$$\nonterm{numeric-const} ::= \mbox{\nonterm{integer-const} $|$
  \nonterm{char-const}}$$
%

\bfhead{Indexed expressions.} An indexed expression consists of a
base expression, an offset expression, and a size:
%
$$\nonterm{indexed-expr} ::= \mbox{\nonterm{expr} \kwd{[}
    \nonterm{expr} \kwd{,} \nonterm{size} \kwd{]}}$$
%

\bfhead{Register expressions.} A register expression consists of a
caret character followed by a 6502 register name:
%
$$\nonterm{register-expr} ::= \mbox{\kwd{\^} \group{\kwd{A} $|$
    \kwd{X} $|$ \kwd{Y} $|$ \kwd{P} $|$ \kwd{S}}}$$
%

\bfhead{Call expressions.} A call expression consists of an expression
followed by an argument list enclosed in parentheses:
%
$$\nonterm{call-expr} ::= \mbox{\nonterm{expr} \kwd{(}
  \opt{\nonterm{expr} \group{\kwd{,} \nonterm{expr}}$^*$} \kwd{)}}$$
%


\bfhead{Set expressions.} A set expression consists of the keyword
\kwd{SET}, and lvalue expression, an equals sign, and a
right-hand-side expression:
%
$$\nonterm{set-expr} ::= \mbox{\kwd{SET} \nonterm{lvalue-expr} \kwd{=}
  \nonterm{expr}}$$
%

\bfhead{Binary operation expressions.} A binary operation expression
consists of a left-hand-side expression, a binary operator, and a
right-hand-side expression:
%
$$\nonterm{binop-expr} ::= \mbox{\nonterm{expr} \nonterm{binop} \nonterm{expr}}$$
%
$$\nonterm{binop} ::= \mbox{\kwd{>} $|$ \kwd{<} $|$ \kwd{<=} $|$
  \kwd{>=} $|$ \kwd{AND} $|$ \kwd{OR} $|$ \kwd{XOR} $|$ \kwd{+} $|$
  \kwd{-} $|$ \kwd{*} $|$ \kwd{/} $|$ \kwd{<<} $|$ \kwd{>>} $|$
  \kwd{=}}$$
%
\bfhead{Unary operation expressions.} A unary operation expression
consists of a unary operator followed by an expression:
%
$$\nonterm{binop-expr} ::= \mbox{\nonterm{unop} \nonterm{expr}}$$
%
$$\nonterm{unop} ::= \mbox{\kwd{@} $|$ \kwd{NOT} $|$ \kwd{-} $|$
  \kwd{INCR} $|$ \kwd{DECR}}$$
%
\bfhead{Parentheses expressions.} A parentheses expression is an
expression surrounded by parentheses:
%
$$\nonterm{parens-expr} ::= \mbox{\kwd{(} \nonterm{expr} \kwd{)}}$$
%

\section{Semantics}
\label{sec:semantics}

TODO

\subsection{Compilation and Linking}
\label{sec:semantics:compilation}

TODO

\begin{comment}
If no backslash character appears after the second double-quote, then
the string is null-terminated:  in addition to storing the characters
appearing in the double-quotes, it stores a terminating NUL character
(value \$00).  If a backslash character appears after the second
double-quote, then the string is unterminated:  only the characters
appearing between the double-quotes are stored.
\end{comment}

\end{document}
